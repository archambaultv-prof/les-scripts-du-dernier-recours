% Compile avec :
% latexmk -lualatex -shell-escape examen.tex

\documentclass[letterpaper, 12pt]{article}

\usepackage{fontspec}          % Needed for LuaLaTeX fonts
\usepackage[french]{babel}     % Needed for French language support
\usepackage{geometry}          % Needed to set page margins
\usepackage{graphicx}          % Needed for \includegraphics (logo)
\usepackage{xcolor}            % Used by minted and TikZ (for colors)
\usepackage{newunicodechar}    % Used to define the epicene dot
\usepackage{enumitem}          % Used for customizing the enumerate environments
\usepackage{minted}            % Used for Python code blocks
\usepackage{tikz}              % Used for custom diagrams (class diagrams)
\usetikzlibrary{arrows.meta, positioning}  % Needed for arrow styling and positioning in TikZ
\usepackage{fancyhdr}          % Used for headers and footers in the document
\usepackage{multicol}          % Used for multiple column layout in the multiple-choice questions
\usepackage{tikz-cd}           % Used for the category theory commutative diagram


\geometry{margin=2.5cm}

% Commande pour point médian épicène
\newunicodechar{·}{\kern-.25em\lower.23ex\hbox{\textperiodcentered}\kern-.25em}

% Command to draw a letter inside a circle
\newcommand{\mcletter}[1]{%
  \tikz[baseline=(char.base)]{%
    \node[shape=circle,draw,inner sep=2pt,minimum size=1.4em,align=center] (char) {#1};%
  }%
}

\setlength{\headheight}{14.5pt}

\begin{document}

% PAGE DE TITRE
\begin{titlepage}

    \begin{center}

    \includegraphics{logo_maisonneuve.png} 
    \vspace{1cm}

    {\Huge\bfseries 420-123-MA : Examen 1\par}
    \vspace{1cm}

    {\large\bfseries 1\textsuperscript{er} avril 2025\par}
    \vfill

    \end{center}

    \begin{description}[leftmargin=0pt]
        \item[Cours] 420-123-MA Programmation hivernale.
        \item[Professeur] Seg Fault.
        \item[Pondération] 15\,\% de la note finale.
        \item[Durée] 120 minutes.
        \item[Documentation] Une feuille de notes manuscrite, recto-verso au
        format lettre.
        \item[Intelligence artificielle] Non permise.
    \end{description}

    \vfill

    \noindent
    {\large\bfseries {Nom de l’étudiant·e}} : \hrulefill

    \vfill

    \noindent
    {\large\bfseries Instructions\par}
    \begin{itemize}
        \item L'examen est composé de 5 questions, pour un total de 100 points.
        \item Les questions sont indépendantes. Vous pouvez faire l'examen dans
        l'ordre de votre choix.
        \item Des pages blanches sont fournies à la fin de l'examen si besoin. Vous pouvez
        les utiliser si vous manquez de place pour répondre à une question.
    \end{itemize}

    \vfill
\end{titlepage}

\newpage

\setcounter{page}{2}
\pagestyle{fancy}
\fancyhf{}
\cfoot{\thepage}
\lhead{420-123-MA — Examen 1}


\section{Types et expressions (20 pts)}

\subsection{Print (10 pts)}
Écrivez ce que va afficher le code suivant.

\begin{minted}[fontsize=\small, linenos]{python}
a = 5

def f(x):
    return x + 1

print(f(a))
\end{minted}

\vfill

\subsection{Types (10 pts)}
Donnez le type de la variable \texttt{a} dans le code suivant.
\begin{minted}[fontsize=\small, linenos]{python}
def foo(x: int) -> int:
    a = x + 1
    return a
\end{minted}

\vfill

\newpage

\section{Fonctions (20 pts)}

Écrivez une fonction qui prend toute la page !

\newpage

\section{Diagramme de classes (20 pts)}

Le digramme de classes ci-dessous est moyen, mais c'est un début.
Ouvrez un pull request pour corriger le code TikZ!

\begin{center}
    \begin{tikzpicture}[
        class/.style={
            rectangle, draw=black, fill=gray!10,
            text width=5cm, minimum height=2cm,
            inner sep=5pt, align=left
        },
        arrow/.style={
            -{Latex[length=3mm]}, thick
        }
    ]
    
    % Livre class
    \node[class] (livre) {
        \textbf{Livre} \\
        \textit{- titre : str} \\
        \textit{- auteur : str} \\
        \rule{\linewidth}{0.4pt} \\
        \texttt{+ \_\_init\_\_(titre, auteur)} \\
        \texttt{+ afficher()}
    };
    
    % Bibliotheque class
    \node[class, right=4cm of livre] (biblio) {
        \textbf{Bibliothèque} \\
        \textit{- livres : list} \\
        \rule{\linewidth}{0.4pt} \\
        \texttt{+ \_\_init\_\_()} \\
        \texttt{+ ajouter(livre)} \\
        \texttt{+ lister()}
    };
    
    % Association
    \draw[arrow] (biblio.west) -- node[above]{1..*} (livre.east);
    
    \end{tikzpicture}
    \end{center}

    \newpage

\section{Questions à choix multiples (20 pts)}

Cochez la bonne réponse pour chaque question. Chaque question vaut 4 points.

\begin{enumerate}[label=\textbf{Q\arabic*.}, leftmargin=1.5cm]

  \item Que retourne la fonction \texttt{len("Bonjour")} en Python ?

  \begin{multicols}{2}
  \begin{enumerate}[label=\protect\mcletter{\alph*}, itemsep=0.5em]
    \item 6
    \item 7
    \item \texttt{"Bonjour"}
    \item Erreur
  \end{enumerate}
    \end{multicols}
  \vspace{1em}

  \item Quelle structure permet de répéter un bloc de code ?

  \begin{multicols}{2}
    \begin{enumerate}[label=\protect\mcletter{\alph*}, itemsep=0.5em]
      \item \texttt{if}
      \item \texttt{while}
      \item \texttt{print}
      \item \texttt{return}
    \end{enumerate}
  \end{multicols}

  \vspace{1em}

  \item Laquelle de ces instructions provoque une erreur ?

  \begin{enumerate}[label=\protect\mcletter{\alph*}, itemsep=0.5em]
    \item \texttt{3 + 4}
    \item \texttt{"3" + 4}
    \item \texttt{len("abc")}
    \item \texttt{int("5")}
  \end{enumerate}

  \vspace{1em}

  \item Que fait \texttt{print("a" * 3)} ?

  \begin{enumerate}[label=\protect\mcletter{\alph*}, itemsep=0.5em]
    \item Affiche \texttt{"aaa"}
    \item Affiche \texttt{"a3"}
    \item Provoque une erreur
    \item Affiche \texttt{"a a a"}
  \end{enumerate}

  \vspace{1em}

  \item Quel mot-clé est utilisé pour définir une fonction en Python ?

  \begin{enumerate}[label=\protect\mcletter{\alph*}, itemsep=0.5em]
    \item \texttt{define}
    \item \texttt{function}
    \item \texttt{def}
    \item \texttt{func}
  \end{enumerate}

\end{enumerate}

\section{Théorie des catégories (20 pts)}

Si \(h \circ f = k \circ g\), est-ce que le diagramme suivant commute ? Répondez par vrai ou faux.

\[
\begin{tikzcd}
A \arrow[r, "f"] \arrow[d, "g"'] & B \arrow[d, "h"] \\
C \arrow[r, "k"'] & D
\end{tikzcd}
\]

\newpage

\section*{Page blanche}

\textbf{N'oubliez pas d'indiquer le numéro de la question.}

\newpage

\section*{Page blanche}

\textbf{N'oubliez pas d'indiquer le numéro de la question.}


\end{document}
