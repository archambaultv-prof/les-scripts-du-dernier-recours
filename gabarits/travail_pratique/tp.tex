% Compile avec :
% latexmk -lualatex -shell-escape tp.tex

\documentclass[letterpaper, 12pt]{article}

\usepackage{fontspec}          % Needed for LuaLaTeX fonts
\usepackage[french]{babel}     % Needed for French language support
\usepackage{geometry}          % Needed to set page margins
\usepackage{graphicx}          % Needed for \includegraphics (logo)
\usepackage[table]{xcolor}            % Used by minted and TikZ and tables (for colors)
\usepackage{newunicodechar}    % Used to define the epicene dot
\usepackage{enumitem}          % Used for customizing the enumerate environments
\usepackage{minted}            % Used for Python code blocks
\usepackage{fancyhdr}          % Used for headers and footers in the document
\usepackage[hidelinks, colorlinks=true, linkcolor=blue,
  urlcolor=blue, citecolor=blue, filecolor=blue,
  pdfborderstyle={/S/U/W 1}]{hyperref} % Used for hyperlinks
\usepackage{pdflscape}         % Used for landscape pages
\usepackage{caption}           % Used for captions in tables
\usepackage{array}             % Used for custom column types in tables
\usepackage{colortbl}          % Pour la commande \columncolor
\usepackage{soul}              % Used for highlighting text

\geometry{margin=2.5cm}

% Commande pour point médian épicène
\newunicodechar{·}{\kern-.25em\lower.23ex\hbox{\textperiodcentered}\kern-.25em}

\setlength{\headheight}{14.5pt}

\newcommand{\boldred}[1]{\textbf{\textcolor{red}{#1}}}

\begin{document}

% PAGE DE TITRE
\begin{titlepage}

    \begin{center}

    \includegraphics{logo_maisonneuve.png} 
    \vspace{1cm}

    {\Huge\bfseries 420-123-MA : Travail pratique 1\par}
    \vspace{0.75cm}

    {\LARGE\bfseries Grille de correction\par}
    \vspace{0.75cm}

    {\large\bfseries 1\textsuperscript{er} avril 2025\par}
    \vfill

    \end{center}

    \begin{description}[leftmargin=0pt]
        \item[Cours] 420-123-MA Pédagogie hivernale.
        \item[Professeur] Tax Bloom
        \item[Pondération] 15\% de la note finale.
        \item[Intelligence artificielle] Autorisée selon la
        \href{https://gitlab.com/infocmaisonneuve/general/-/blob/master/PDEA_INFORMATIQUE.pdf}
        {politique du département}.
        \item[Date de remise] 15 avril 23h59.
        \item[Mode de remise] Remise électronique via Omnivox.
    \end{description}

    \vfill
    \vfill

\end{titlepage}

\newpage

\setcounter{page}{2}
\pagestyle{fancy}
\fancyhf{}
\cfoot{\thepage}
\lhead{420-123-MA — Travail partique 1}


\section{Introduction}

Ce document est un gabarit pour des énoncés de travaux pratiques. Pour joindre
l'utile à l'agréable, il contient aussi les connaissances de base sur les
grilles de correction. Un grand merci à \href{https://www.polymtl.ca/expertises/farand-patrice}
{Patrice Farand}, professeur à Polytechnique, pour son cours sur l'évaluation
en enseignement supérieur.

Les grilles de correction sont un outil essentiel pour évaluer le travail des
étudiants. Elles permettent de définir des critères d'évaluation clairs et
objectifs et de fournir un retour constructif sur le travail effectué. Elles
sont également utilisées par les étudiants pour s'auto-évaluer et s'assurer
qu'ils répondent aux attentes du professeur. De plus, elles permettent de
réduire les biais de l'évaluateur en fournissant une structure claire pour
l'évaluation.

Ainsi, de part sa structure, les grilles de correction économisent le temps de
l'évaluateur tout en garantissant que l'évaluation est juste et équitable. Dans
ce document, nous allons présenter un modèle de grille de correction.

\section{Modèle de grille de correction}

La grille de correction décrite dans ce document présente les critères
d'évaluation de manière concise. Elle permet de visualiser rapidement les
différents niveaux de performance pour chaque critère. Chaque case de la grille
contient une description du niveau de performance attendu pour le critère
correspondant.

Il est important d'énoncer le critère d'évaluation à l'aide d'un verbe d'action
mesurable. Par exemple, au lieu d'écrire « Polymorphisme », on
pourrait dire « Appliquer correctement le polymorphisme ». Le verbe d'action
doit être en lien avec le niveau taxonomique de la compétence que l'on souhaite
évaluer. Vous pouvez consulter 
\href{https://www.enseigner.ulaval.ca/system/files/public/pedagogie/preparer-votre-cours/taxonomie-de-bloom-revisee.pdf}
{ce document} de l'université Laval pour des exemples de verbes d'action.

\subsection{Lexique de la grille de correction}

La table ci-dessous illustre le lexique utilisé pour les grilles d'évaluation.
Le nombre de niveau peut varier selon le type de travail à évaluer.
\begin{table}[ht]
  \centering
  \captionsetup{position=top}
  \caption{Lexique utilisé pour les grilles d'évaluations}
  \newcolumntype{R}[2]{>{\raggedright\arraybackslash\columncolor{#2}}p{#1}}
  \fontsize{10pt}{12pt}\selectfont
  % \newlength{\extracolsepwidth}
  % \setlength{\extracolsepwidth}{\dimexpr 2\tabcolsep\relax}
  \begin{tabular}
    {R{2cm}{gray!10}R{2cm}{green!30}R{2cm}{green!30!yellow!40}R{2cm}{yellow!40}R{2cm}{orange!40}R{2cm}{red!40}}
    
    &
    \textbf{Échelle} \newline Niveau 1 &
    \textbf{Échelle} \newline  Niveau 2  &
    \textbf{Échelle} \newline  Niveau 3  &
    \textbf{Échelle} \newline Niveau 4   &
    \textbf{Échelle} \newline  Niveau 5  \\
    \hline
    \textbf{Critère} & & & & & \\
    Indicateur et dimension &
    Descripteur &
    Descripteur &
    Descripteur &
    Descripteur &
    Descripteur \\
  \end{tabular}
\end{table}

\subsection{Exemple de grille de correction}

Un exemple de grille de correction est présenté à la page suivante. Il s'agit
d'une grille d'évaluation pour évaluer les grilles d'évaluation ! Elle est bien
sûr incomplète. 

\newpage
\renewcommand{\arraystretch}{1.2} % Increase row spacing by a factor
\begin{landscape} % Begin landscape mode
  \begin{table}[ht]
    \centering
    \captionsetup{position=top}
    \caption{Grille d’évaluation pour grille d'évaluation (1/2)}
    \label{tab:grille}
    \newcolumntype{R}[2]{>{\raggedright\arraybackslash\columncolor{#2}}p{#1}}
    \fontsize{10pt}{12pt}\selectfont
    \newlength{\extracolsepwidth}
    \setlength{\extracolsepwidth}{\dimexpr 2\tabcolsep\relax}
    \begin{tabular}
      {R{3.5cm}{gray!10}R{3.5cm}{green!30}R{3.5cm}{green!30!yellow!40}R{3.5cm}{yellow!40}R{3.5cm}{orange!40}R{3.5cm}{red!40}}
      
       &
      \textbf{Excellent} \newline $\geq$ 90\% &
      \textbf{Très bien} \newline $\geq$ 80\% &
      \textbf{Bien} \newline $\geq$ 70\% &
      \textbf{Passable} \newline $\geq$ 60\% &
      \textbf{Insuffisant} \newline $<$ 60\% \\
      \hline
      \textbf{Formuler des critères mesurables (25\%)} & & & & & \\
    Verbe d'action mesurable à l'infinitif &
    Tous les critères sont décrits à l'aide d'un verbe d'action mesurable à l'infinitif. &
    Tous les critères, sauf un, sont décrits à l'aide d'un verbe  d'action mesurable à l'infinitif. &
    La majorité des critères sont décrits à l'aide d'un verbe d'action mesurable  à l'infinitif. &
    La moitié des critères sont décrits à l'aide d'un verbe  d'action mesurable à l'infinitif. &
    Une minorité des critères sont décrits à l'aide d'un verbe  d'action mesurable à l'infinitif. \\

    Niveau taxonomique adéquat &
    Tous les critères sont décrits à l'aide d'un verbe en lien avec le niveau taxonomique de la compétence. &
    Tous les critères sauf un sont décrits à l'aide d'un verbe en lien avec le niveau taxonomique de la compétence. &
    La majorité des critères sont décrits à l'aide d'un verbe en lien avec le niveau taxonomique de la compétence. &
    La moitié des critères sont décrits à l'aide d'un verbe en lien avec le niveau taxonomique de la compétence. &
    Une minorité des critères sont décrits à l'aide d'un verbe en lien avec le niveau taxonomique de la compétence. \\

    \textbf{Choisir une échelle appropriée (25\%)} & & & & & \\
    Nombre de niveaux adéquat &
    
    L'échelle est en adéquation avec le nombre de critères et d'indicateurs,
    ainsi que le travail demandé. Le nombre de niveau permet au professeur de
    les distinguer facilement. &

    L'échelle est généralement en adéquation avec les critères, les indicateurs
    et le travail demandé. Le nombre de niveau permet au professeur de
    les distinguer assez facilement. &
    
    L'échelle correspond partiellement aux critères et au travail demandé. Le
    nombre de niveaux est parfois suffisant, mais il arrive que le professeur
    hésite. &

    L'échelle correspond faiblement aux critères et au travail demandé. Le
    nombre de niveaux est rarement suffisant, il arrive souvent que le
    professeur hésite. &
    
    L'échelle ne reflète pas clairement les critères ou le travail demandé. Le
    nombre de niveaux est limité ou trop grand et ne permet pas de faire des
    distinctions significatives. \\
  
  \end{tabular}
\end{table}
\end{landscape}

% FIXME : Maybe use longtable ???
\newpage

\begin{landscape} % Begin landscape mode
  \begin{table}[ht]
    \centering
    \captionsetup{position=top}
    \caption{Grille d’évaluation pour grille d'évaluation (2/2)}
    \label{tab:grille_2}
    \newcolumntype{R}[2]{>{\raggedright\arraybackslash\columncolor{#2}}p{#1}}
    \fontsize{10pt}{12pt}\selectfont
    \begin{tabular}
      {R{3.5cm}{gray!10}R{3.5cm}{green!30}R{3.5cm}{green!30!yellow!40}R{3.5cm}{yellow!40}R{3.5cm}{orange!40}R{3.5cm}{red!40}}
      
        &
      \textbf{Excellent} \newline $\geq$ 90\% &
      \textbf{Très bien} \newline $\geq$ 80\% &
      \textbf{Bien} \newline $\geq$ 70\% &
      \textbf{Passable} \newline $\geq$ 60\% &
      \textbf{Insuffisant} \newline $<$ 60\% \\
      \hline

  \textbf{Utiliser des indicateurs avec dimension (25\%)} & & & & & \\
  Dimension précisée &

  Tous les indicateurs sont précis et communiquent clairement la dimension
  évaluée. Cette dernière est facilement appréciable. &

  Tous les indicateurs, sauf un, sont précis et communiquent clairement la
  dimension évaluée. Cette dernière est appréciable sans grande difficulté. &

  La majorité des indicateurs sont précis et communiquent clairement la
  dimension évaluée. Cette dernière est appréciable, parfois avec difficulté. &

  La moitié des indicateurs sont précis et communiquent clairement la
  dimension évaluée. Cette dernière est peu appréciable. &

  Une minorité des indicateurs sont précis et communiquent clairement la
  dimension évaluée. Cette dernière est difficilement
  appréciable par le professeur. \\

  Quantité adéquate &

  Les indicateurs sont juste assez nombreux pour évaluer facilement l'atteinte
  du critère. Ils ne sont pas redondants. &

  Les indicateurs sont généralement en nombre suffisant pour évaluer le
  critère. Certains pourraient être perçus comme légèrement redondants
  ou moins pertinents. &

  Les indicateurs clés sont présents, mais d'autres sont superflus ou
  redondants. &
  
  Certains indicateurs clés sont absents. D'autres sont superflus ou
  redondants.&

  Les indicateurs sont mal choisis, trop nombreux ou inutiles. Leur redondance
  et leur flou compromettent sérieusement l’évaluation du critère.\\

  \textbf{Écrire un descripteur précis (25\%)} & & & & & \\
  Énoncé clair et constructif &

  Le descripteur dirige parfaitement l'élève vers le résultat attendu.
  L'énoncé est court, explicite et univoque. La modulation est progressive
  et nuancée. &

  Le descripteur dirige somme toute bien l'élève vers le résultat attendu.
  L'énoncé est court, explicite et univoque, sauf pour un détail mineur.
  La modulation est progressive et nuancée, sauf pour un détail mineur. &

  Le descripteur dirige l'élève vers le résultat attendu, mais pourrait être
  plus précis. L'énoncé est plus ou moins court, explicite et univoque. La
  modulation manque de nuance ou progresse un peu trop rapidement ou trop
  lentement. &

  Le descripteur dirige peu l'élève vers le résultat attendu.
  L'énoncé est devrait être réécrit, mais reste compréhensible. La modulation est trop rapide ou
  trop lente. &

  Le descripteur dirige très peu l'élève vers le résultat attendu.
  L'énoncé est trop long, vague ou ambigu. La modulation est beaucoup trop rapide ou
  trop lente. \\


  \end{tabular}
  \end{table}
\end{landscape}

\clearpage
\subsection{Rétroaction avec une grille d’évaluation}

Un exemple de rétroaction donnée à l'aide la grille de la
table~\ref{tab:grille} est fourni à la page suivante. Notez qu'il est possible,
pour un même indicateur, que plusieurs descripteurs soient partiellement
atteints.

Une bonne rétroaction devrait aussi s'accompagner d'une séries de commentaires
constructifs. L'objectif est d'aider l'élève à se situer quant à son atteinte
des compétences visées par le cours et, le cas échéant, de lui indiquer comment
y arriver.

D'ailleurs, ce sont les commentaires qui permettront d'expliquer pourquoi deux élèves
ayant obtenu le même niveau de performance dans la grille de correction n'ont pas
nécessairement exactement la même note.

\begin{landscape} % Begin landscape mode
  \begin{table}[ht]
    \centering
    \captionsetup{position=top}
    \caption{Grille d’évaluation pour grille d'évaluation (1/2)}
    \label{tab:grille_retroaction}
    \newcolumntype{R}[2]{>{\raggedright\arraybackslash\columncolor{#2}}p{#1}}
    \fontsize{10pt}{12pt}\selectfont
    \begin{tabular}
      {R{3.5cm}{gray!10}R{3.5cm}{green!30}R{3.5cm}{green!30!yellow!40}R{3.5cm}{yellow!40}R{3.5cm}{orange!40}R{3.5cm}{red!40}}
      
      \boldred{Note finale: 87,5\%} &
      \textbf{Excellent} \newline $\geq$ 90\% &
      \textbf{Très bien} \newline $\geq$ 80\% &
      \textbf{Bien} \newline $\geq$ 70\% &
      \textbf{Passable} \newline $\geq$ 60\% &
      \textbf{Insuffisant} \newline $<$ 60\% \\
      \hline
      \textbf{Formuler des critères mesurables (25\%)} & &  \boldred{85\%}& & & \\
    Verbe d'action mesurable à l'infinitif &
    Tous les critères sont décrits à l'aide d'un verbe d'action mesurable à l'infinitif. &
    \hl{Tous les critères, sauf un, sont décrits à l'aide d'un verbe  d'action mesurable à l'infinitif.} &
    La majorité des critères sont décrits à l'aide d'un verbe d'action mesurable  à l'infinitif. &
    La moitié des critères sont décrits à l'aide d'un verbe  d'action mesurable à l'infinitif. &
    Une minorité des critères sont décrits à l'aide d'un verbe  d'action mesurable à l'infinitif. \\

    Niveau taxonomique adéquat &
    Tous les critères sont décrits à l'aide d'un verbe en lien avec le niveau taxonomique de la compétence. &
    \hl{Tous les critères sauf un sont décrits à l'aide d'un verbe en lien avec le niveau taxonomique de la compétence.} &
    La majorité des critères sont décrits à l'aide d'un verbe en lien avec le niveau taxonomique de la compétence. &
    La moitié des critères sont décrits à l'aide d'un verbe en lien avec le niveau taxonomique de la compétence. &
    Une minorité des critères sont décrits à l'aide d'un verbe en lien avec le niveau taxonomique de la compétence. \\

    \textbf{Choisir une échelle appropriée (25\%)} & \boldred{100\%} & & & & \\
    Nombre de niveaux adéquat &
    
    \hl{L'échelle est en adéquation avec le nombre de critères et d'indicateurs,
    ainsi que le travail demandé. Le nombre de niveau permet au professeur de
    les distinguer facilement.} &

    L'échelle est généralement en adéquation avec les critères, les indicateurs
    et le travail demandé. Le nombre de niveau permet au professeur de
    les distinguer assez facilement. &
    
    L'échelle correspond partiellement aux critères et au travail demandé. Le
    nombre de niveaux est parfois suffisant, mais il arrive que le professeur
    hésite. &

    L'échelle correspond faiblement aux critères et au travail demandé. Le
    nombre de niveaux est rarement suffisant, il arrive souvent que le
    professeur hésite. &
    
    L'échelle ne reflète pas clairement les critères ou le travail demandé. Le
    nombre de niveaux est limité ou trop grand et ne permet pas de faire des
    distinctions significatives. \\
  
  \end{tabular}
\end{table}
\end{landscape}

\begin{landscape} % Begin landscape mode
  \begin{table}[ht]
    \centering
    \captionsetup{position=top}
    \caption{Grille d’évaluation pour grille d'évaluation (2/2)}
    \label{tab:grille_retroaction_2}
    \newcolumntype{R}[2]{>{\raggedright\arraybackslash\columncolor{#2}}p{#1}}
    \fontsize{10pt}{12pt}\selectfont
    \begin{tabular}
      {R{3.5cm}{gray!10}R{3.5cm}{green!30}R{3.5cm}{green!30!yellow!40}R{3.5cm}{yellow!40}R{3.5cm}{orange!40}R{3.5cm}{red!40}}
      
        &
      \textbf{Excellent} \newline $\geq$ 90\% &
      \textbf{Très bien} \newline $\geq$ 80\% &
      \textbf{Bien} \newline $\geq$ 70\% &
      \textbf{Passable} \newline $\geq$ 60\% &
      \textbf{Insuffisant} \newline $<$ 60\% \\
      \hline

  \textbf{Utiliser des indicateurs avec dimension (25\%)} & \boldred{90\%} & & & & \\
  Dimension précisée &

  \hl{Tous les indicateurs sont précis et communiquent clairement la dimension
  évaluée. Cette dernière est facilement appréciable.} &

  Tous les indicateurs, sauf un, sont précis et communiquent clairement la
  dimension évaluée. Cette dernière est appréciable sans grande difficulté. &

  La majorité des indicateurs sont précis et communiquent clairement la
  dimension évaluée. Cette dernière est appréciable, parfois avec difficulté. &

  La moitié des indicateurs sont précis et communiquent clairement la
  dimension évaluée. Cette dernière est peu appréciable. &

  Une minorité des indicateurs sont précis et communiquent clairement la
  dimension évaluée. Cette dernière est difficilement
  appréciable par le professeur. \\

  Quantité adéquate &

  \hl{Les indicateurs sont juste assez nombreux pour évaluer facilement l'atteinte
  du critère.} Ils ne sont pas redondants. &

  Les indicateurs sont généralement en nombre suffisant pour évaluer le
  critère. \hl{Certains pourraient être perçus comme légèrement redondants
  ou moins pertinents.} &

  Les indicateurs clés sont présents, mais d'autres sont superflus ou
  redondants. &
  
  Certains indicateurs clés sont absents. D'autres sont superflus ou
  redondants.&

  Les indicateurs sont mal choisis, trop nombreux ou inutiles. Leur redondance
  et leur flou compromettent sérieusement l’évaluation du critère.\\

  \textbf{Écrire un descripteur précis (25\%)} & & & \boldred{75\%} & & \\
  Énoncé clair et constructif &

  Le descripteur dirige parfaitement l'élève vers le résultat attendu.
  L'énoncé est court, explicite et univoque. La modulation est progressive
  et nuancée. &

  \hl{Le descripteur dirige somme toute bien l'élève vers le résultat attendu.}
  L'énoncé est court, explicite et univoque, sauf pour un détail mineur.
  La modulation est progressive et nuancée, sauf pour un détail mineur. &

  Le descripteur dirige l'élève vers le résultat attendu, mais pourrait être
  plus précis. \hl{L'énoncé est plus ou moins court, explicite et univoque. La
  modulation manque de nuance ou progresse un peu trop rapidement ou trop
  lentement.} &

  Le descripteur dirige peu l'élève vers le résultat attendu.
  L'énoncé devrait être réécrit, mais reste compréhensible. La modulation est trop rapide ou
  trop lente. &

  Le descripteur dirige très peu l'élève vers le résultat attendu.
  L'énoncé est trop long, vague ou ambigu. La modulation est beaucoup trop rapide ou
  trop lente. \\


  \end{tabular}
  \end{table}

\clearpage
\section{Conclusion}

Il est fortement conseillé de distribuer la grille de correction aux étudiants, par exemple en annexe
de l'énoncé du travail. Non seulement cela va les guider dans leur travail, mais cela va aussi grandement
vous faciliter la tâche de l'évaluation. Un travail bien fait et en adéquation avec la grille de
correction est souvent un travail qui se corrige facilement. Cela limite aussi les recours possibles
pour une correction jugée injuste ou arbitraire.

\subsection{Pour plus de plaisir}

Consulter le dépôt href{https://github.com/archambaultv-prof/les-scripts-du-dernier-recours}
{\texttt{les-scripts-du-dernier-recours}} pour d'autres gabarits et scripts.

\end{landscape}

\end{document}
